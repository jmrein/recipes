\documentclass[../dinner.tex]{subfiles}
\begin{document}
	\pagestyle{fancy}
	\fancyhf{}
	\newpage
	\section{Peanut stew}
	\lhead{}\chead{Serves 1}\rhead{V}
	\lfoot{Prep time: 5m}\rfoot{Cook time: 30m}
	{\Large Ingredients}
	\begin{multicols}{2}
		\begin{itemize}
			\item \(\frac{1}{2}\) onion
			\item 1 garlic clove
			\item 1 tbsp peanut butter
			\item \(\frac{1}{2}\) tin chopped tomatoes
			\item \(\frac{1}{2}\) tin kidney beans
			\item 1 green pepper
		\end{itemize}
		\columnbreak
		\begin{itemize}
			\item \(\frac{1}{2}\) vegetable stock cube
			\item 75g couscous
			\item 1 tsp ground ginger
			\item 2 tsp chilli powder
			\item 1 tsp cinnamon
		\end{itemize}
	\end{multicols}
	
	{\Large Instructions}
	\begin{enumerate}
		\item Peel and finely chop the onion and garlic. Remove the core from the pepper and chop finely.
		\item Boil a kettle. Put the couscous in a bowl with a little stock powder and salt, mix together, then add enough boiled water to just cover it. Cover with a tea towel and leave to the side.
		\item Heat 1 tbsp oil in a pan on medium-low heat, fry the onions until soft (5 minutes), add the garlic and cook for another two minutes.
		\item Add in the chopped green pepper and cook for four minutes. Add 200ml boiled water and the \(\frac{1}{2}\) stock cube. Warm the mixture through and stir in the peanut butter. Stir until it melts and disperses through the mixture.
		\item Add in the chopped tomatoes and rinse the black beans and add them too. Stir through. Mix in the ginger, cinnamon and half the chilli powder. Cook gently for 5 minutes and then add the remaining chilli powder plus salt \& pepper to taste.
		\item Let the mixture simmer down. Fluff up the couscous with a fork and then serve the mixture over a bed of the couscous.
	\end{enumerate}
\end{document}