\documentclass[../dinner.tex]{subfiles}
\begin{document}
	\pagestyle{fancy}
	\fancyhf{}
	\newpage
	\section{Winter veg risotto}
	\lhead{}\chead{Serves 1}\rhead{V}
	\lfoot{Prep time: 10m}\rfoot{Cook time: 30m}
	{\Large Ingredients}
	\begin{multicols}{2}
		\begin{itemize}
			\item \(\frac{1}{2}\) onion
			\item \(\frac{1}{4}\) swede
			\item \(\frac{1}{2}\) parsnip
			\item \(\frac{1}{2}\) carrot
			\item \(\frac{1}{2}\) stalk celery
			\item 1 clove garlic
		\end{itemize}
		\columnbreak
		\begin{itemize}
			\item 100g arborio rice
			\item 1 vegetable stock pot
			\item \(\frac{1}{2}\) tbsp butter
			\item (optional) a little white wine
			\item small bunch of parsley
			\item a little hard cheese
		\end{itemize}
	\end{multicols}

	{\Large Instructions}
	\begin{enumerate}
		\item Peel and finely dice the onion and garlic. Dice the swede, parsnip, carrot and celery. Grate the cheese and roughly chop the parsley.
		\item Bring 400ml of water to a gentle simmer and stir in the stock cube
		\item Heat \(\frac{1}{2}\) tbsp of butter and \(\frac{1}{2}\) tbsp olive oil in a pot on medium-low heat. Add the onion, garlic, swede, parsnip, carrot and celery and cook for 5 minutes.
		\item Add the arborio rice and coat in the oil and butter. Turn heat to medium and add a pinch of salt. After about 3 minutes the edges of the rice should be a little translucent.
		\item Add a few glugs of wine if available, let it bubble off for a minute. If not, skip to the next step.
		\item Add \(\frac{1}{3}\) cup of stock to the rice and stir in with smooth motions. Continue stirring slowly, but don't overdo it unless you want textureless dinner porridge.
		\item Once the stock is almost soaked in, add another \(\frac{1}{3}\) cup of stock. Continue adding stock (and stirring) like this until the rice is cooked (about 20 minutes). If you run out of stock add some extra water as needed until the rice is soft (but still textured).
		\item Stir in the cheese and half the parsley. Check for taste and add salt \& pepper if needed. Serve topped with the remaining parsley.
	\end{enumerate}
\end{document}